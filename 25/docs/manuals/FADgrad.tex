\documentclass[12pt,letterpaper]{article}
\usepackage{amsmath}
\usepackage{amsfonts}
\usepackage{color}
\usepackage{graphicx}
\usepackage[pdftex,bookmarksopen]{hyperref}
\hypersetup{pdfauthor={John Sibert}}
\hypersetup{pdfsubject={tagest User's Manual}}
\hypersetup{pdftitle={tagest Users Manual}}
\hypersetup{pdfkeywords={tagest, tagmove, jnigraphics, tag attrition,
ADRM}}
\hypersetup{pdfpagemode=UseThumbs}

%\markright{Sibert \hfil tagest User's Manual\hfil \today}

\newcommand\doublespacing{\baselineskip=1.6\normalbaselineskip}
\newcommand\singlespacing{\baselineskip=1.0\normalbaselineskip}
\renewcommand\deg[1]{$^\circ$#1}
\newcommand\uvD{$(uvD)$}
\newcommand\SD{SEAPODYM}
\newcommand\SDTE{SEAPODYM-TAGEST}
\newcommand\UK{uKfSST}
\newcommand\EK{ensemble Kalman filter}
\newcommand\ADMB{ADModel Builder} \title{\TT\ User's Manual}
\newcommand\help[1]{\color{red}{\it #1 }\normalcolor}
\newcommand\TT{{\tt tagest \& tagmove}}
\newcommand\TE{{\tt tagest}}
\newcommand\TM{{\tt tagmove}}
\newcommand\PAR{{\tt .par}}
\newcommand\TAG{{\tt .tag}}
\newcommand\ADRM{Advection Diffusion Reaction Model}

\author{
John Sibert\thanks{sibert@hawaii.edu}\\
Joint Institute of Marine and Atmospheric Research\\
University of Hawai'i at Manoa\\
Honolulu, HI  96822 U.S.A.\\[0.125in]
\date{\today}
}

\title{Using FAD gradients to compute advection in \TT}

\begin{document}
\maketitle

\doublespacing

Let $[z]$ mean ``the unit (or dimension) of the variable $z$''.
The unit of the advective components, $u$ and $v$, of movement in \TT\
is nautical miles per month or nm mo$^{-1}$; 
$[u] = $ nm mo$^{-1}$.
(Confusingly, $(u,v)$ are reported in
nm da$^{-1}$ in the \PAR\ files and converted to nm mo$^{-1}$ in {\tt
uvcomp.cpp}.) 

Let $G$ be the density of FADs;  $[G] = $ FADs nm$^{-2}$. We are
interested in relating the horizontal gradients of FAD density to
advection  by a simple function such that $u = f(\frac{dG}{dx})$;
$[\frac{dG}{dx}] = $ FADs nm$^{-2}$nm$^{-1} = $ FADs nm$^{-3}$. Here
are three possible formulations for $f(\frac{dG}{dx})$.

1. Simple linear relationship between advection and
FAD gradient 
\[
u = C\cdot \frac{dG}{dx}
.\] 
What is [C]? 
We can write $[u] = [C]\cdot [\frac{dG}{dx}]$ and solve for $[C]$.
\[
[C] = \frac{[u]}{[\frac{dG}{dx}]} 
    = \frac{{\rm nm\; mo}^{-1}}{\rm{FADs\;nm}^{-3}}
.
\]
%   = \frac{{\rm mo}^{-1}}{\rm{FADs\;nm}^{-2}}
%   = \frac{1}{{\rm FADs}}\cdot\frac{{\rm nm}^{2}}{\rm{mo}}
%   = \frac{{\rm nm}}{{\rm FADs}}\cdot\frac{{\rm nm}}{{\rm mo}}
Any way you try to simplify the units, $C$ is an awkward constant.

2. The current model assumes slightly different linear relationship
\[
u = C \cdot s^* \cdot \frac{dG}{dx},
\]
where $s^*$ is the maximum sustained swimming speed;
$[s^*] = {\rm nm\; mo}^{-1}$.
Then
\[
[C] = \frac{[u]}{[s^*][\frac{dG}{dx}]} 
    = \frac{{\rm nm\; mo}^{-1}}{{\rm nm\; mo}^{-1}\cdot\rm{FADs\;nm}^{-3}}
    = \frac{1}{\rm{FADs\;nm}^{-3}}
,\]
which appears to directly relate advection to the gradient of FAD
density.

3. It is possible to ``normalize'' the FAD density to the mean FAD
density. Let $G^* = G/\bar{G}$, where $\bar{G}$ is the mean FAD
density; $[\bar{G}] = {\rm{FADs\;nm}^{-2}}$.
Then $\frac{dG^*}{dx} = \frac{1}{\bar{G}}\cdot\frac{dG}{dx}$.


\[
u = C \cdot \frac{s^*}{\bar{G}} \cdot \frac{dG}{dx},
\]
where $s^*$ is the maximum sustained swimming speed as before.
Then
\[
[C] = \frac{[u]}{\frac{[s^*]}{[\bar{G}]}\cdot[\frac{dG}{dx}]} 
    = \frac{[u]\cdot[\bar{G}]}{[s^*]\cdot[\frac{dG}{dx}]} 
    = \frac{{\rm nm\; mo}^{-1}\cdot{\rm{FADs\;nm}^{-2}}}{{\rm nm\; mo}^{-1}\cdot\rm{FADs\;nm}^{-3}}
    = \frac{1}{\rm{FADs\;nm}^{-1}}
.\]

It is hard to say which of these three alternatives are best. I was
hoping for a dimensionless coefficient, but couldn't find it easily.
You should check my math though. I sort of like the third option
because the dimensions are simpler and the normalization gets avoid
some problems of interpreting the weighting factor as a measure of
regional sensitivity.

\end{document}
